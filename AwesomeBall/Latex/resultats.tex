\chapter{Description de votre stratégie de validation}

\section{Utilisation d'un package test}

Pour tester les points importants de notre jeu, nous avons créé plusieurs classes "test" dans un package mis à l'écart. Voici le contenu du package "test" :

\paragraph*{ClientTest.java :}
Cette classe sert à détecter les erreurs de réception et/ou d'envoi de chaînes de caractères au serveur.  

\paragraph*{ExitButtonTest.java :} Elle sert à vérifier si le bouton "Exit" lorsque le joueur 2 quitte la partie fonctionne.

\paragraph*{FieldTest.java :} Elle sert à détecter des erreurs lors de l'initialisation du Field, en liant le contrôle au visuel, utilisant du JUnit Testing

\paragraph*{InetTest.java :} Elle vérifie la détection de l'IP local par java (et savoir, par la même occasion, l'IP local du joueur lançant cette application).


\paragraph*{SwingWorkerTest :} Elle teste la recherche d'IP local (donc la classe DiscoverLocal.java).

\paragraph*{TextFieldTest :} Cette classe, utilisant le JUnit Testing, vérifie si l'affichage textuelle sur le terrain fonctionne. 

\section{JUnit Testing :}

On a utilisé le JUnit Testing principalement pour l'affichage textuelle sur le terrain ( via la classe TextFieldTest) ainsi que pour l'initialisation du terrain, en liant le contrôle au visuel (via la classe FieldTest)

\section{System.out.println}

Cette méthode a été utilisé très fréquemment pour détecter les collisions entre les objets (balle, terrain et joueur) .